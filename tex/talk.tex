\documentclass[xcolor=dvipsnames,compress,aspectratio=169]{beamer}
\usepackage{beamerthemeCS-Saar-Uni_sic}
\usepackage[utf8]{inputenc}
\usepackage[english]{babel}
\usepackage[OT1]{fontenc}
\usepackage{hyperref}
\usepackage{graphicx}
\usepackage{amsfonts,amsmath}
\usepackage{cleveref}
\usepackage{scalerel}
\usepackage{multicol}
\usepackage{xspace}
\usepackage{amsmath}
\usepackage{amssymb}
\usepackage{stmaryrd}
\usepackage{proof}
\usepackage{natbib}

\usepackage{tikz}
\usetikzlibrary{calc}
\usetikzlibrary{arrows.meta}
\def\checkmark{\tikz\fill[scale=0.4](0,.35) -- (.25,0) -- (1,.7) -- (.25,.15) -- cycle;}

\title[GCH Implies AC in HoTT]{The Generalised Continuum Hypothesis\\Implies the Axiom of Choice in HoTT}
\author[\underline{Dominik Kirst} and Felix Rech]{\underline{Dominik Kirst} and Felix Rech}
\date[HoTT/UF'21]{Workshop on Homotopy Type Theory / Univalent Foundations\\July 18, 2021}

\newcommand{\nologo}{\setbeamertemplate{logo}{}} % command to set the logo to nothing


\DeclareMathAlphabet{\mymathbb}{U}{bbold}{m}{n}
\DeclareMathOperator*{\dbigvee}{\scalerel*{\bigvee}{\sum}}

% BOXES
\usepackage[many]{tcolorbox}
\newtcolorbox{emptybox}[1][]{
	beamer,
	width=(0.7\textwidth),
	% enlarge left by=-3pt,
	titlerule=0mm,
	colframe=white,
	coltitle=black,
	bottom=6pt, 
	top=-12pt,
	left=6pt,  
	right=6pt,
	notitle, 
	adjusted title={},
	outer arc=.5mm,
	arc=.5mm,
	no shadow,
	fuzzy shadow={1mm}{-1mm}{-1.2mm}{.7mm}{black!20},
	interior titled code={}
}

\newtcolorbox{widerbox}[1][]{
	beamer,
	width=(0.82\textwidth),
	% enlarge left by=-3pt,
	titlerule=0mm,
	colframe=white,
	coltitle=black,
	bottom=6pt, 
	top=-12pt,
	left=6pt,  
	right=6pt,
	notitle, 
	adjusted title={},
	outer arc=.5mm,
	arc=.5mm,
	no shadow,
	fuzzy shadow={1mm}{-1mm}{-1.2mm}{.7mm}{black!20},
	interior titled code={}
}

\newtcolorbox{widebox}[1][]{
	beamer,
	width=(\textwidth),
	% enlarge left by=-3pt,
	titlerule=0mm,
	colframe=white,
	coltitle=black,
	bottom=6pt, 
	top=-12pt,
	left=6pt,  
	right=6pt,
	notitle, 
	adjusted title={},
	outer arc=.5mm,
	arc=.5mm,
	no shadow,
	fuzzy shadow={1mm}{-1mm}{-1.2mm}{.7mm}{black!20},
	interior titled code={}
}


% TIMELINE
\usepackage{tikz}
\usepackage{graphicx}
\usepackage{stackengine}
\usetikzlibrary{trees, shapes, calc, positioning, shadows}

\definecolor{darkcerulean}{rgb}{0.063661, 0.257392, 0.477463}
\definecolor{scooter}{rgb}{0.161162, 0.775760, 0.885416}


% #1 – box name
% #2 – left color
% #3 – right color
% #4 – Title (may contain #1)
\newcommand{\newlargebox}[4]{
	\newtcolorbox{#1}[1][]{
		beamer,
		width=\textwidth+7pt,
		enlarge left by=-3pt,
		titlerule=3mm,
		colframe=white,
		coltitle={#2},
		bottom=2pt, 
		top=-4pt,
		left=6pt,  
		toptitle=2pt,
		bottomtitle=-2pt,
		fonttitle=\bfseries\large,
		adjusted title={#4},
		outer arc=.5mm,
		arc=.5mm,
		no shadow,
		fuzzy shadow={1mm}{-1mm}{-1.2mm}{.7mm}{black!20},
		interior titled code={
			\path [left color = {#2}, right color = {#3}]
			(title.south west) + (8pt, 0) rectangle ++(\textwidth-1pt, 0.02);
		}
	}
}


\newlargebox{defaultbox}{darkcerulean}{scooter}{#1} 

\usepackage{fancyvrb}
\usepackage{xcolor}


%%% fonts in math environment
\newcommand{\MBB}[1]{\ensuremath{\mathbb{#1}}\xspace}  % blackboard
\newcommand{\MBF}[1]{\ensuremath{\mathbf{#1}}\xspace}  % bold
\newcommand{\MCL}[1]{\ensuremath{\mathcal{#1}}\xspace} % calligraphy
\newcommand{\MTT}[1]{\ensuremath{\mathtt{#1}}\xspace}  % type writer
\newcommand{\MRM}[1]{\ensuremath{\mathrm{#1}}\xspace}  % plain
\newcommand{\MIT}[1]{\ensuremath{\mathit{#1}}\xspace}  % italic
\newcommand{\MSF}[1]{\ensuremath{\mathsf{#1}}\xspace}  % sans


%%% common symbols
\newcommand{\Nat}{\MBB{N}}   % The natural numbers
\newcommand{\Int}{\MBB{Z}}   % The whole numbers
\newcommand{\Rtn}{\MBB{Q}}   % The rational numbers
\newcommand{\Real}{\MBB{R}}  % The real numbers
\newcommand{\Bool}{\MBB{B}}  % The booleans
\newcommand{\Unit}{\mymathbb{1}}  % Unit type



%%% marcros
\newcommand{\Pow}{\MCL P}
\newcommand{\Prop}{\MBB P}
\newcommand{\Type}{\MBB T}
\newcommand{\btrue}{\MSF{tt}}
\newcommand{\bfalse}{\MSF{ff}}
\newcommand{\ordinals}{\MCL O}
\newcommand{\hartogsNumber}{\aleph}
\newcommand{\succs}[1]{\sigma(#1)}
\newcommand{\rec}{\MSF{rec}}
\newcommand{\Set}{\MCL S}
\newcommand{\WF}{\MCL W}
\newcommand{\GCHS}{\textnormal{GCH}_\Set}
\newcommand{\WOS}{\textnormal{WO}_\Set}
\newcommand{\ACS}{\textnormal{AC}_\Set}
\newcommand{\GCHT}{\textnormal{GCH}_\Type}
\newcommand{\WOT}{\textnormal{WO}_\Type}
\newcommand{\ACT}{\textnormal{AC}_\Type}
\newcommand{\GCHR}{\textnormal{GCH}_\Prop}
\newcommand{\ACR}{\textnormal{AC}_\Prop}


\begin{document}

%\beamerdefaultoverlayspecification{<+->}
\renewcommand{\emph}[1]{\textcolor{sb@mcyan}{#1}}
\definecolor{red}{RGB}{204,0,0}
\definecolor{yellow}{RGB}{228,242,31}
\definecolor{green}{RGB}{0,204,0}

\newcommand\refs[1]{%
	\begin{textblock*}{8cm}(0.3cm,9.2cm)%
		\scriptsize {\color{gray}#1}
	\end{textblock*}
}


%\begin{frame}
%  \tableofcontents
%\end{frame}

\nologo
\begin{frame}
	\maketitle
\end{frame}

\let\footnoterule\relax

\begin{frame}{The Generalised Continuum Hypothesis Implies the Axiom of Choice in ZF}
	\pause
	\begin{centering}
		\begin{widerbox}
			\center
			There are no cardinalities between an infinite set and its power set (GCH)\\
			\vspace{0.1cm}
			$\Downarrow$\\
			\vspace{0.1cm}
			Every set has a choice function / can be well-ordered (AC/WO)
		\end{widerbox}
	\end{centering}

	\pause
	\vspace{0.2cm}
	\begin{itemize}
		\item
		Result in ZF set theory announced by~\cite{lindenbaum_communication_1926}
		\vspace{0.1cm}
		\item
		First published proof by~\cite{sierpinski1947}
		\vspace{0.1cm}
		\item
		Refinement using GCH more locally by~\cite{specker_verallgemeinerte_1990}
		\pause
		\vspace{0.2cm}
		\item
		Mechanisation in Metamath by~\cite{carneiro_gch_2015}
		\vspace{0.1cm}
		\item
		Paper ``GCH implies AC in Coq'' by~\cite{kirst2021generalised}\footnote<4->{Mostly following \cite{gillman2002} and \cite{smullyan2010}.}
		\begin{itemize}
			\item
			Two mechanised variants: higher-order ZF and Coq's type theory
		\end{itemize}
	\end{itemize}
\end{frame}

\begin{frame}{Set Theory in Coq's Type Theory}
	%Some abstract set-theoretic results apply to dependent type theories,\\ e.g. the equivalence of WO and AC ( \cite{ilik_zermelo,smolka_transfinite_2015}).
	\vspace{0.2cm}
	Using impredicative universe $\Prop$ and propositional existence $(\exists x.\,P\,x):\Prop$ we have:
	
	\vspace{0.2cm}
	\begin{center}
	\begin{tabular}{c|c|c}
				&ZF set theory&Coq's Type Theory\\\hline
				\vphantom{\vdots}Membership&$x\in y$&$x:X~$ (for $X:\Type$)\\[0.2cm]
				Power sets&$\Pow(A)$&$X\to\Prop$\\[0.2cm]
				Numbers&$\omega$&$\Nat$\\[0.2cm]
				%Relations&$\Pow(A\times B)$&$X\to Y\to\Prop$\\[0.2cm]
				%Functions&$\{f\subseteq A\times B\mid \dots \}$&$X\to Y$\\[0.2cm]
				Cardinality&$\exists f\subseteq A\times B\dots$&$\exists f:X\to Y\dots$\\[0.2cm]
				Orderings&$\exists R\subseteq A\times A\dots$&$\exists R:X\to X\to\Prop\dots$
	\end{tabular}
	\end{center}
	
	\pause
	\vspace{0.2cm}
	Axioms necessary to make Coq's type theory behave like set theory:
	\begin{itemize}
	\item Functional extensionality, to tame function space
	\item Propositonal extensionality, to tame predicate space
	\item Unique choice, to identify functions with total functional relations
	\end{itemize}
\end{frame}

\newcommand{\trunc}[1]{||#1||}
\newcommand{\hprop}{\mathsf{hProp}}
\newcommand{\hset}{\mathsf{hSet}}
\newcommand{\Ord}{\mathsf{Ord}}

\begin{frame}{Set Theory in Homotopy Type Theory}
	\vspace{0.2cm}
	Using propositional resizing to represent propositions in $\Omega:\mathcal{U}_0$ we have:

	\vspace{0.2cm}
	\begin{center}
	\begin{tabular}{c|c|c}
				&ZF set theory&Homotopy Type Theory\\\hline
				\vphantom{\vdots}Membership&$x\in y$&$x:X~$ (for $X:\hset$)\\[0.2cm]
				Power sets&$\Pow(A)$&$X\to\Omega$\\[0.2cm]
				Numbers&$\omega$&$\Nat$\\[0.2cm]
				%Relations&$\Pow(A\times B)$&$X\to Y\to\Prop$\\[0.2cm]
				%Functions&$\{f\subseteq A\times B\mid \dots \}$&$X\to Y$\\[0.2cm]
				Cardinality&$\exists f\subseteq A\times B\dots$&$\trunc{\Sigma f:X\to Y\dots}$\\[0.2cm]
				Orderings&$\exists R\subseteq A\times A\dots$&$\trunc{\Sigma R:X\to X\to\Omega\dots}$
	\end{tabular}
	\end{center}
	
	\pause
	\vspace{0.2cm}
	Naturally suited to represent set theory:
	\begin{itemize}
	\item Functional extensionality: implied by univalence
	\item Propositonal extensionality: implied by univalence
	\item Unique choice: by the elimination principle of propositional truncation
	\end{itemize}
	
%	\pause
%	\vspace{0.3cm}
%	Over-approximates ZF: equipotent sets are equal (doesn't affect structural result)
\end{frame}

\begin{frame}{Formulating GCH in HoTT}
	With $X\le Y$ as propositional cardinality comparison $\trunc{\Sigma f:X\to Y.\,\mathsf{injective}\,f}$:
	
	\pause
	\vspace{0.5cm}
	\begin{centering}
		\begin{widerbox}
			\center
			There are no cardinalities between an infinite set and its power set.
			$$
			\only<3>{\forall X Y:\hset.~\textcolor{lightgray}{\Nat \le X\le Y\le \Pow(X)~\,\to~\, Y\le X ~+~ \Pow(X)\le Y}}
			\only<4>{\textcolor{lightgray}{\forall X Y:\hset.}~\Nat \le X\textcolor{lightgray}{~\le Y\le \Pow(X)~\,\to~\, Y\le X ~+~ \Pow(X)\le Y}}
			\only<5>{\textcolor{lightgray}{\forall X Y:\hset.~\Nat \le~} X\le Y\le \Pow(X)\textcolor{lightgray}{~\,\to~\, Y\le X ~+~ \Pow(X)\le Y}}
			\only<6>{\textcolor{lightgray}{\forall X Y:\hset.~\Nat \le X\le Y\le \Pow(X)}~\,\to~\, Y\le X ~+~ \Pow(X)\le Y}
			\only<7->{\forall X Y:\hset.~\Nat \le X\le Y\le \Pow(X)~\,\to~\, Y\le X ~+~ \Pow(X)\le Y}
			$$
			%			$$\only<2>{
			%			\forall X Y.\, |\Nat|\le |X|\to \textcolor{lightgray}{|X|  \le |Y|  \le |\Pow(X)|\to |Y|\le|X| \lor |\Pow(X)|\le |Y|}}
			%			\only<3>{
			%			\textcolor{lightgray}{\forall X Y.\, |\Nat|\le |X|\to~} |X|  \le |Y|  \le |\Pow(X)|\to \textcolor{lightgray}{|Y|\le|X| \lor |\Pow(X)|\le |Y|}}
			%			\only<4>{
			%			\textcolor{lightgray}{\forall X Y.\, |\Nat|\le |X|\to |X|  \le |Y|  \le |\Pow(X)|\to ~}|Y|\le|X| \lor |\Pow(X)|\le |Y|}
			%			\only<5->{
			%			\forall X Y.\, |\Nat|\le |X|\to |X|  \le |Y|  \le |\Pow(X)|\to |Y|\le|X| \lor |\Pow(X)|\le |Y|}$$
		\end{widerbox}
	\end{centering}
	
	\pause\pause\pause\pause\pause
	\vspace{0.5cm}
	\begin{itemize}
	 	\pause
		\item
		Proposition since concluding disjunction is exclusive (Cantor's theorem)
		\vspace{0.3cm}
		\pause
		\item
		Formulated positively since cardinalities aren't comparable without AC
		\vspace{0.3cm}
		\pause
		\item
		Conclusion just the missing comparison, not yet the equivalence
	\end{itemize}
\end{frame}

\begin{frame}{GCH implies LEM}
	Already a weaker formulation of CH$~=~$GCH($\Nat$) implies the excluded middle (LEM):
	\begin{fact}[cf. \cite{bridges2016continuum}]
		$(\forall X:\hset.\,\Nat \le X\le \Pow(\Nat)\to X\le \Nat+ \Pow(\Nat)\le X)~\to~\forall P:\hprop.\,P+\neg P$
	\end{fact}
	\pause
	\begin{proof}
		\begin{enumerate}
			\pause
			\item
			Given $P:\hprop$, the set $X:=\Sigma\, p:\Pow(\Nat).\,\trunc{\mathsf{singleton}\,p+ (P+\neg P)}$ satisfies the premises.
			\pause
			\item
			We can even show $X\not\le \Nat$, hence we obtain an injection $i:\Pow(\Nat)\to X$.
			\pause
			\item
			By a variant of Cantor's theorem there is $p:\Pow(\Nat)$ such that $\pi_1(i\,p)$ is not a singleton.
			\pause
			\item
			Thus $P+\neg P$ must be the case.
			\qedhere
		\end{enumerate}
	\end{proof}
	\pause
	So by classical reasoning, i.e.\ the Cantor-Bernstein theorem:
	\begin{corollary}
		GCH is equivalent to $\forall X Y:\hset.\,\Nat \le X\le Y\le \Pow(X)~\to~ Y = X \,+\, Y=\Pow(X) $.
	\end{corollary}
\end{frame}

\begin{frame}
	\Huge
	\centering
	Proof Overview
\end{frame}

\begin{frame}{Outline}
	\begin{enumerate}[<+->]
			\item
			Instead of AC, show the equivalent WO
			\vspace{0.4cm}
			\item
			To well-order $X$ it suffices to find ordinal $\alpha$ with $X\le \alpha$
%			\vspace{0.4cm}
%			\item
%			Enough to only well-order infinite $X$ since always $X\le\Nat + X$
			\vspace{0.4cm}
			\item
			Central construction: Hartogs number $\hartogsNumber(X)$
			\begin{itemize}
				\item
				\vspace{0.2cm}
				Large ordinal: $\hartogsNumber(X)\not\le X$
				\vspace{0.2cm}
				\item
				Controlled height: $\hartogsNumber(X)\le\Pow^3(X)$
			\end{itemize}
			\vspace{0.4cm}
			\item
			Develop cardinal arithmetic in the absence of AC
			\vspace{0.4cm}
			\item
			Use GCH to iteratively squeeze in $\hartogsNumber(X)$ and obtain $X\le \hartogsNumber(X)$
	%		\begin{itemize}
	%			\vspace{0.3cm}
	%			\item
	%			Suitable instance yields $|X|\le|H(X)|$ or $|H(X)|\le |\Pow^{k-1}(X)|$
	%			\vspace{0.3cm}
	%			\item
	%			Iterate in the latter case until $|H(X)|\le |\Pow^{0}(X)|=|X|$
	%		\end{itemize}
		\end{enumerate}
\end{frame}

\begin{frame}{Constructive Ordinal Numbers (Chapter 10.3 of the HoTT book)}
	\begin{definition}
		An ordinal is a set equipped with a well-founded, extensional, transitive, mere relation.
	\end{definition}

	\pause
	\vspace{0.5cm}
	Properties needed for main result:
	\begin{itemize}
		\vspace{0.2cm}
		\item
		Isomorphic ordinals are equal (instance of SIP)
		\vspace{0.2cm}
		\item
		Type $\Ord$ of ordinals with natural ordering is an ordinal
		\vspace{0.2cm}
		\item
		Every ordinal is isomorphic to its set of initial segments
		\vspace{0.2cm}
		\item
		Ordinals satisfy trichotomy and have least elements (requiring LEM)
	\end{itemize}

	\pause
	\vspace{0.5cm}
	Also successor and limit ordinals mechanised but irrelevant for main result.
\end{frame}

\begin{frame}{The Hartogs Number $\aleph(X)$ of a Set $X$, without AC}
	\pause
	\begin{definition}
		We define $\aleph'(X):\Ord$ as the type of ordinals $\alpha$ with $\alpha \le X$, ordered by the natural ordering.
	\end{definition}

	\pause
	\vspace{0.3cm}
	The ordinal $\aleph'(X)$ lives in a higher universe level than $X$, therefore need to resize:
	\pause
	\begin{theorem}
		Using LEM, we obtain $\aleph(X)$ by resizing $\aleph'(X)$ along the canonical injection $\aleph'(X)\le\Pow^3(X)$.
		Then $\aleph(X)$ is in the same universe as $X$ and satisfies $\aleph(X)\le\Pow^3(X)$ as well as $\aleph(X)\not\le X$.
	\end{theorem}
	\pause
	\begin{proof}
		\begin{enumerate}
			\pause
			\item
			Injection $i: \aleph'(X)\to\Pow^3(X)$ maps $\alpha\le X$ to its induced order on $X$ (using trichotomy).
			\pause
			\item
			Obtain $\aleph(X)$ as range of $i$ with ordering of $\aleph'(X)$, then $\aleph(X)\le\Pow^3(X)$ by construction.
			\pause
			\item
			$\aleph(A)\not\le A$ since otherwise $\aleph(A)$ would be an initial segment of the isomorphic $\aleph'(A)$.
			\qedhere
		\end{enumerate}
	\end{proof}
\end{frame}

\begin{frame}{Cardinal Arithmetic, without AC}

With AC, infinite sets $X$ satisfy $X\simeq X+X$.
\pause
Without AC we get:

\begin{lemma}
	Using LEM, every set $X$ with $\Nat\le X$ satisfies $X\simeq\Unit+X$ and $\Pow(X)\simeq\Pow(X)+\Pow(X)$.
\end{lemma}
\vspace{-0.1cm}
\pause
\begin{proof}[Sketch]
	By equational reasoning, e.g. the former implies the latter as follows:
	$\Pow(X)\overset{\textnormal{LEM}}\simeq~\Pow(\Unit+X)~\simeq~\Pow(\Unit)\times \Pow (X)\overset{\textnormal{LEM}}\simeq~\Bool\times \Pow(X)~\simeq~ \Pow(X)+\Pow(X)$
\end{proof}

\pause
\vspace{0.3cm}
Call $X$ \emph{large enough} if $X\simeq X+X$, then using Cantor's theorem once again:
\begin{lemma}
	For sets $X$ large enough and $Y$ with $\Pow (X)\le X+Y$ we obtain $\Pow (X)\le Y$.
\end{lemma}
\pause
\vspace{-0.1cm}
\begin{proof}[Sketch]
	Obtain $i:\Pow (X)\times\Pow (X)\hookrightarrow X+Y$, use $\lambda p.\, i(p,c):\Pow (X)\hookrightarrow Y$, $c$ the diagonal set of $i^{-1}$.
\end{proof}

\end{frame}

\begin{frame}{Iterate GCH to conclude $X\le \aleph(X)$}

	\pause
	\begin{theorem}
		Assume GCH and a function $F:\hset_i\to\hset_i$ such that there is $k:\Nat$ with $F(X)\le \Pow^k(X)$ and $F(X)\not\le X$ for all $X$.
		Then for every large enough set $X$ we obtain $X\le F(X)$.
	\end{theorem}
	\vspace{-0.1cm}
	\pause
	\begin{proof}
		\pause
		Given a large enough set $X$, we show $X\le F(X)$ by induction on k:
		\begin{itemize}
			\vspace{0.2cm}
			\pause
			\item
			If $k=0$ the assumptions $F(X)\le \Pow^k(X)$ and $F(X)\not\le X$ are contradictory.
			\vspace{0.2cm}
			\pause
			\item
			For $k+1$ apply GCH to the situation $\Pow^{k}(X)~\le~ \Pow^{k}(X)+F(X)~\le~ \Pow^{k+1}(X)$:
			\begin{itemize}
			\vspace{0.2cm}
			\pause
			\item
			If $\Pow^{k}(X)+F(X)\le \Pow^{k}(X)$ then already $F(X)\le \Pow^{k}(X)$, conclude with IH.
			\vspace{0.2cm}
			\item
			\pause
			If $\Pow^{k+1}(X)\le \Pow^{k}(X)+F(X)$ then already $\Pow^{k+1}(X)\le F(X)$, conclude $X\le F(X)$.
			\qedhere
			\end{itemize}
		\end{itemize}
	\end{proof}
	\pause
	\vspace{-0.1cm}
	\begin{corollary}
		GCH implies AC.
	\end{corollary}
\end{frame}

\begin{frame}
	\Huge
	\centering
	Observations
\end{frame}

\begin{frame}{Mechanisation Details}

	\pause
	\vspace{0.25cm}
			Based on (and contributed to) the Coq HoTT library (\cite{bauer2017hott})
			\begin{itemize}
				\vspace{0.15cm}
				\item
				Cardinals, ordinals, Hartogs numbers, GCH $\to$ LEM, GCH $\to$ AC, 5 versions of Cantor
				\vspace{0.15cm}
				\item
				1400 lines in total (1300 relevant for result, 700 on ordinals, 250 on Hartogs number)
				\vspace{0.15cm}
				\item
				Some code from previous development could be reused
				\vspace{0.15cm}
				\item
				Easy to work with for pure Coq users (tactics, notations, typeclass for $\hprop$)
			\end{itemize}

	\pause
	\vspace{0.5cm}
	Only difficulties connected to power sets and universes:
	\begin{itemize}
		\vspace{0.15cm}
		\item
		Resizing by hand tedious and sometimes very slow
		\vspace{0.15cm}
		\item
		Power sets actually defined as $X\to\hprop$, only resized where needed
		\vspace{0.15cm}
		\item
		Construction of $\aleph(X)$ in two parts for performance reasons
		\vspace{0.15cm}
		\item
		Showing that power sets are sets caused universe conflicts with section usage
	\end{itemize}
\end{frame}

\begin{frame}{Comparison to Previous Coq Mechanisation}
	\pause
	\begin{itemize}
		\item
		Similar proofs concerning cardinal arithmetic and main theorem
		\begin{itemize}
			\item Some necessary equivalences already in HoTT library
		\end{itemize}
		\vspace{0.2cm}
		\pause
		\item
		Ordinals admit a better organized set-theoretic construction of $\aleph(X)$
		\begin{itemize}
			\item Previous development based on ``small'' ordinals embeddable into $X$
		\end{itemize}
		\vspace{0.2cm}
		\pause
		\item
		Setoid rewriting could be avoided by univalence
		\begin{itemize}
			\item No need for morphism lemmas like $|X|=|Y|\to |\Pow(X)|=|\Pow(Y)|$
		\end{itemize}
		\vspace{0.2cm}
		\pause
		\item
		Overall code reduction from 1700loc to 1300loc relevant for main theorem
		\begin{itemize}
			\item Code for $\aleph(X)$ more principled than previous ad-hoc construction
		\end{itemize}
		\vspace{0.2cm}
		\pause
		\item
		More satisfying theoretical foundation for set-theoretic results
		\begin{itemize}
			\item
			Could dispose of the ad-hoc assumptions from previous development
		\end{itemize}
		\vspace{0.2cm}
		\pause
		\item
		Caveat: Coq users are used to static impredicativity
		\begin{itemize}
			\item
			Hard to trace and debug implicitly added universe constraints
		\end{itemize}
	\end{itemize}
\end{frame}

\begin{frame}{Open Questions}
	\begin{centering}
		\begin{widerbox}
		\center
		Could a static $\mathsf{HProp}$ universe be used instead of resizing by hand?
		\end{widerbox}
		\pause
		\vspace{0.3cm}
		\begin{widerbox}
		\center
		Is ``light HoTT" combining weak univalence with UIP in static $\Prop$ useful?
		\end{widerbox}
		\pause
		\vspace{0.3cm}
		\begin{widerbox}
		\center
		Are there meaningful/consistent formulations of GCH for higher n-types?
		\end{widerbox}
		\pause
		\vspace{0.3cm}
		\begin{widerbox}
		\center
		Do constructive versions of GCH imply constructive versions of WO?
		\end{widerbox}
	\end{centering}
\end{frame}



%\begin{frame}[allowframebreaks]{Related Work}
\begin{frame}{Bibliography}
\footnotesize
%\small
\bibliographystyle{apalike}
\bibliography{abstract}
\end{frame}

\begin{frame}{Variants of Cantor's theorem}
\begin{fact}[Injective Cantor]
	Given a type $X$, there is no injection $\Pow(X)\le X$.
\end{fact}
\vspace{-0.1cm}
\begin{fact}[Singleton Cantor]
	Given a set $X$ and an injection $i:\Pow(X)\to\Pow(X)$, there is $p$ s.t.\ $i\,p$ is not a singleton.
\end{fact}
\vspace{-0.1cm}
\begin{fact}[Surjective Cantor]
	Given a type $X$ and a function $f:X\to \Pow(X)$, there is $p$ s.t.\ $f\,x\not = p$ for all $x$.
\end{fact}
\vspace{-0.1cm}
\begin{fact}[Predicative Cantor]
	Given a type $X$ and a function $f:X\to (X \to \mathcal U)$, there is $p$ s.t.\ $f\,x\not = p$ for all $x$.
\end{fact}
\vspace{-0.1cm}
\begin{fact}[Relational Cantor]
	Given a type $X$ and a functional relation $R:X\to \Pow(X)\to\Omega$, there is $p$ s.t.\ $\neg R\,x\,p$ for all $x$.
\end{fact}
\end{frame}


\end{document}

