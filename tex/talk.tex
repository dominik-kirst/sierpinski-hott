\documentclass[xcolor=dvipsnames,compress,aspectratio=169,handout]{beamer}
\usepackage{beamerthemeCS-Saar-Uni_sic}
\usepackage[utf8]{inputenc}
\usepackage[english]{babel}
\usepackage[OT1]{fontenc}
\usepackage{hyperref}
\usepackage{graphicx}
\usepackage{amsfonts,amsmath}
\usepackage{cleveref}
\usepackage{scalerel}
\usepackage{multicol}
\usepackage{xspace}
\usepackage{amsmath}
\usepackage{amssymb}
\usepackage{stmaryrd}
\usepackage{proof}
\usepackage{natbib}

\usepackage{tikz}
\usetikzlibrary{calc}
\usetikzlibrary{arrows.meta}
\def\checkmark{\tikz\fill[scale=0.4](0,.35) -- (.25,0) -- (1,.7) -- (.25,.15) -- cycle;}

\title[GCH Implies AC in HoTT]{The Generalised Continuum Hypothesis\\Implies the Axiom of Choice in HoTT}
\author[\underline{Dominik Kirst} and Felix Rech]{\underline{Dominik Kirst} and Felix Rech}
\date[HoTT/UF'21]{Workshop on Homotopy Type Theory / Univalent Foundations\\July 18, 2021}

\newcommand{\nologo}{\setbeamertemplate{logo}{}} % command to set the logo to nothing


\DeclareMathAlphabet{\mymathbb}{U}{bbold}{m}{n}
\DeclareMathOperator*{\dbigvee}{\scalerel*{\bigvee}{\sum}}

% BOXES
\usepackage[many]{tcolorbox}
\newtcolorbox{emptybox}[1][]{
	beamer,
	width=(0.7\textwidth),
	% enlarge left by=-3pt,
	titlerule=0mm,
	colframe=white,
	coltitle=black,
	bottom=6pt, 
	top=-12pt,
	left=6pt,  
	right=6pt,
	notitle, 
	adjusted title={},
	outer arc=.5mm,
	arc=.5mm,
	no shadow,
	fuzzy shadow={1mm}{-1mm}{-1.2mm}{.7mm}{black!20},
	interior titled code={}
}

\newtcolorbox{widerbox}[1][]{
	beamer,
	width=(0.82\textwidth),
	% enlarge left by=-3pt,
	titlerule=0mm,
	colframe=white,
	coltitle=black,
	bottom=6pt, 
	top=-12pt,
	left=6pt,  
	right=6pt,
	notitle, 
	adjusted title={},
	outer arc=.5mm,
	arc=.5mm,
	no shadow,
	fuzzy shadow={1mm}{-1mm}{-1.2mm}{.7mm}{black!20},
	interior titled code={}
}

\newtcolorbox{widebox}[1][]{
	beamer,
	width=(\textwidth),
	% enlarge left by=-3pt,
	titlerule=0mm,
	colframe=white,
	coltitle=black,
	bottom=6pt, 
	top=-12pt,
	left=6pt,  
	right=6pt,
	notitle, 
	adjusted title={},
	outer arc=.5mm,
	arc=.5mm,
	no shadow,
	fuzzy shadow={1mm}{-1mm}{-1.2mm}{.7mm}{black!20},
	interior titled code={}
}


% TIMELINE
\usepackage{tikz}
\usepackage{graphicx}
\usepackage{stackengine}
\usetikzlibrary{trees, shapes, calc, positioning, shadows}

\definecolor{darkcerulean}{rgb}{0.063661, 0.257392, 0.477463}
\definecolor{scooter}{rgb}{0.161162, 0.775760, 0.885416}


% #1 – box name
% #2 – left color
% #3 – right color
% #4 – Title (may contain #1)
\newcommand{\newlargebox}[4]{
	\newtcolorbox{#1}[1][]{
		beamer,
		width=\textwidth+7pt,
		enlarge left by=-3pt,
		titlerule=3mm,
		colframe=white,
		coltitle={#2},
		bottom=2pt, 
		top=-4pt,
		left=6pt,  
		toptitle=2pt,
		bottomtitle=-2pt,
		fonttitle=\bfseries\large,
		adjusted title={#4},
		outer arc=.5mm,
		arc=.5mm,
		no shadow,
		fuzzy shadow={1mm}{-1mm}{-1.2mm}{.7mm}{black!20},
		interior titled code={
			\path [left color = {#2}, right color = {#3}]
			(title.south west) + (8pt, 0) rectangle ++(\textwidth-1pt, 0.02);
		}
	}
}


\newlargebox{defaultbox}{darkcerulean}{scooter}{#1} 

\usepackage{fancyvrb}
\usepackage{xcolor}


%%% fonts in math environment
\newcommand{\MBB}[1]{\ensuremath{\mathbb{#1}}\xspace}  % blackboard
\newcommand{\MBF}[1]{\ensuremath{\mathbf{#1}}\xspace}  % bold
\newcommand{\MCL}[1]{\ensuremath{\mathcal{#1}}\xspace} % calligraphy
\newcommand{\MTT}[1]{\ensuremath{\mathtt{#1}}\xspace}  % type writer
\newcommand{\MRM}[1]{\ensuremath{\mathrm{#1}}\xspace}  % plain
\newcommand{\MIT}[1]{\ensuremath{\mathit{#1}}\xspace}  % italic
\newcommand{\MSF}[1]{\ensuremath{\mathsf{#1}}\xspace}  % sans


%%% common symbols
\newcommand{\Nat}{\MBB{N}}   % The natural numbers
\newcommand{\Int}{\MBB{Z}}   % The whole numbers
\newcommand{\Rtn}{\MBB{Q}}   % The rational numbers
\newcommand{\Real}{\MBB{R}}  % The real numbers
\newcommand{\Bool}{\MBB{B}}  % The booleans
\newcommand{\Unit}{\mymathbb{1}}  % Unit type



%%% marcros
\newcommand{\Pow}{\MCL P}
\newcommand{\Prop}{\MBB P}
\newcommand{\Type}{\MBB T}
\newcommand{\btrue}{\MSF{tt}}
\newcommand{\bfalse}{\MSF{ff}}
\newcommand{\ordinals}{\MCL O}
\newcommand{\hartogsNumber}{\aleph}
\newcommand{\succs}[1]{\sigma(#1)}
\newcommand{\rec}{\MSF{rec}}
\newcommand{\Set}{\MCL S}
\newcommand{\WF}{\MCL W}
\newcommand{\GCHS}{\textnormal{GCH}_\Set}
\newcommand{\WOS}{\textnormal{WO}_\Set}
\newcommand{\ACS}{\textnormal{AC}_\Set}
\newcommand{\GCHT}{\textnormal{GCH}_\Type}
\newcommand{\WOT}{\textnormal{WO}_\Type}
\newcommand{\ACT}{\textnormal{AC}_\Type}
\newcommand{\GCHR}{\textnormal{GCH}_\Prop}
\newcommand{\ACR}{\textnormal{AC}_\Prop}


\begin{document}

%\beamerdefaultoverlayspecification{<+->}
\renewcommand{\emph}[1]{\textcolor{sb@mcyan}{#1}}
\definecolor{red}{RGB}{204,0,0}
\definecolor{yellow}{RGB}{228,242,31}
\definecolor{green}{RGB}{0,204,0}

\newcommand\refs[1]{%
	\begin{textblock*}{8cm}(0.3cm,9.2cm)%
		\scriptsize {\color{gray}#1}
	\end{textblock*}
}


%\begin{frame}
%  \tableofcontents
%\end{frame}

\nologo
\begin{frame}
	\maketitle
\end{frame}

\begin{frame}{The Generalised Continuum Hypothesis Implies the Axiom of Choice in ZF}
	\begin{centering}
		\begin{widerbox}
			\center
			There are no cardinalities between an infinite set and its power set (GCH)\\
			\vspace{0.1cm}
			$\Downarrow$\\
			\vspace{0.1cm}
			Every set has a choice function / can be well-ordered (AC/WO)
		\end{widerbox}
	\end{centering}

	\vspace{0.3cm}
	\begin{itemize}
		\item
		Result in ZF set theory announced by~\cite{lindenbaum_communication_1926}
		\vspace{0.1cm}
		\item
		First published proof by~\cite{sierpinski1947}
		\vspace{0.1cm}
		\item
		Refinement using GCH more locally by~\cite{specker_verallgemeinerte_1990}
		\vspace{0.3cm}
		\item
		Mechanisation in Metamath by~\cite{carneiro_gch_2015}
		\vspace{0.1cm}
		\item
		``GCH implies AC in Coq'' by~\cite{kirst2021generalised}
		\begin{itemize}
			\item
			Two mechanised variants: higher-order ZF and Coq's type theory
		\end{itemize}
	\end{itemize}
\end{frame}

\begin{frame}{Set Theory in Coq's Type Theory}
	%Some abstract set-theoretic results apply to dependent type theories,\\ e.g. the equivalence of WO and AC ( \cite{ilik_zermelo,smolka_transfinite_2015}).
	Using impredicative universe $\Prop$ and propositional existence $(\exists x.\,P\,x):\Prop$ we have:
	
	\vspace{0.3cm}
	\begin{center}
	\begin{tabular}{c|c|c}
				&ZF set theory&Coq's Type Theory\\\hline
				\vphantom{\vdots}Power sets&$\Pow(A)$&$X\to\Prop$\\[0.2cm]
				Numbers&$\omega$&$\Nat$\\[0.2cm]
				%Relations&$\Pow(A\times B)$&$X\to Y\to\Prop$\\[0.2cm]
				%Functions&$\{f\subseteq A\times B\mid \dots \}$&$X\to Y$\\[0.2cm]
				Cardinality&$\exists f\subseteq A\times B\dots$&$\exists f:X\to Y\dots$\\[0.2cm]
				Orderings&$\exists R\subseteq A\times A\dots$&$\exists R:X\to X\to\Prop\dots$
	\end{tabular}
	\end{center}
	
	\vspace{0.3cm}
	Axioms necessary to make Coq's type theory behave like set theory:
	\begin{itemize}
	\item Functional extensionality, to tame function space
	\item Propositonal extensionality, to tame predicate space
	\item Unique choice, to identify functions with total functional relations
	\end{itemize}
\end{frame}

\newcommand{\trunc}[1]{||#1||}
\newcommand{\hprop}{\mathsf{hProp}}
\newcommand{\hset}{\mathsf{hSet}}
\newcommand{\Ord}{\mathsf{Ord}}

\begin{frame}{Set Theory in Homotopy Type Theory}
	Using propositional resizing to represent propositions in $\Omega:\mathcal{U}_0$ we have:

	\vspace{0.1cm}
	\begin{center}
	\begin{tabular}{c|c|c}
				&ZF set theory&Homotopy Type Theory\\\hline
				\vphantom{\vdots}Power sets&$\Pow(A)$&$X\to\Omega$\\[0.2cm]
				Numbers&$\omega$&$\Nat$\\[0.2cm]
				%Relations&$\Pow(A\times B)$&$X\to Y\to\Prop$\\[0.2cm]
				%Functions&$\{f\subseteq A\times B\mid \dots \}$&$X\to Y$\\[0.2cm]
				Cardinality&$\exists f\subseteq A\times B\dots$&$\trunc{\Sigma f:X\to Y\dots}$\\[0.2cm]
				Orderings&$\exists R\subseteq A\times A\dots$&$\trunc{\Sigma R:X\to X\to\Omega\dots}$
	\end{tabular}
	\end{center}
	
	\vspace{0.1cm}
	Naturally suited to represent set theory:
	\begin{itemize}
	\item Functional extensionality, implied by univalence
	\item Propositonal extensionality, implied by univalence
	\item Unique choice, by the elimination principle of propositional truncation
	\end{itemize}
	
	\vspace{0.3cm}
	Deviation from ZF: equipotent sets are equal (doesn't affect result, simplifies mechanisation)
\end{frame}

\begin{frame}{Formulating GCH in HoTT}
	With $X\le Y$ as propositional cardinality comparison $\trunc{\Sigma f:X\to Y.\,\mathsf{injective}\,f}$:
	
	\vspace{0.5cm}
	\begin{centering}
		\begin{widerbox}
			\center
			There are no cardinalities between an infinite set and its power set.
			$$\forall X Y:\hset.~\Nat \le X\le Y\le \Pow(X)~\,\to~\, Y\le X ~+~ \Pow(X)\le Y$$
			%			$$\only<2>{
			%			\forall X Y.\, |\Nat|\le |X|\to \textcolor{lightgray}{|X|  \le |Y|  \le |\Pow(X)|\to |Y|\le|X| \lor |\Pow(X)|\le |Y|}}
			%			\only<3>{
			%			\textcolor{lightgray}{\forall X Y.\, |\Nat|\le |X|\to~} |X|  \le |Y|  \le |\Pow(X)|\to \textcolor{lightgray}{|Y|\le|X| \lor |\Pow(X)|\le |Y|}}
			%			\only<4>{
			%			\textcolor{lightgray}{\forall X Y.\, |\Nat|\le |X|\to |X|  \le |Y|  \le |\Pow(X)|\to ~}|Y|\le|X| \lor |\Pow(X)|\le |Y|}
			%			\only<5->{
			%			\forall X Y.\, |\Nat|\le |X|\to |X|  \le |Y|  \le |\Pow(X)|\to |Y|\le|X| \lor |\Pow(X)|\le |Y|}$$
		\end{widerbox}
	\end{centering}
	
	\vspace{0.5cm}
	\begin{itemize}
		\item
		Proposition since concluding disjunction is exclusive (Cantor's theorem)
		\vspace{0.3cm}
		\item
		Formulated positively since cardinalities aren't comparable without AC
		\vspace{0.3cm}
		\item
		Conclusion just the missing comparison, not yet the equivalence
	\end{itemize}
\end{frame}

\begin{frame}{GCH implies LEM}
	Already a weaker formulation of CH$~=~$GCH($\Nat$) implies the excluded middle (LEM):
	\begin{lemma}[cf. \cite{bridges2016continuum}]
		$(\forall X:\hset.\,\Nat \le X\le \Pow(\Nat)\to X\not\le \Nat\to \Pow(\Nat)\le X)~\to~\forall P:\hprop.\,P+\neg P$
	\end{lemma}
	\begin{proof}
		\begin{enumerate}
			\item
			Given $P:\hprop$, the set $X:=\Sigma\, p:\Pow(\Nat).\,\mathsf{singleton}\,p\lor (P+\neg P)$ satisfies the premises.
			\item
			Hence we obtain an injection $i:\Pow(\Nat)\to X$.
			\item
			By a variant of Cantor's theorem there is $p:\Pow(\Nat)$ such that $\pi_1(i\,p)$ is not a singleton.
			\item
			Thus $P+\neg P$ must be the case.
			\qedhere
		\end{enumerate}
	\end{proof}
	So by classical reasoning, i.e.\ the Cantor-Bernstein theorem:
	\begin{corollary}
		GCH is equivalent to $\forall X Y:\hset.\,\Nat \le X\le Y\le \Pow(X)~\to~ Y = X \,+\, \Pow(X) = Y$.
	\end{corollary}
\end{frame}

\begin{frame}
	\Huge
	\centering
	Proof Overview
\end{frame}

\begin{frame}{Outline}
	\begin{enumerate}[<+->]
			\item
			Instead of AC, show the equivalent WO
			\vspace{0.4cm}
			\item
			To well-order $X$ it suffices to find ordinal $\alpha$ with $|X|\le |\alpha|$
			\vspace{0.4cm}
			\item
			Enough to only well-order infinite $X$ since always $|X|\le|\Nat + X|$
			\vspace{0.4cm}
			\item
			Central construction: \emph{Hartogs number} $\hartogsNumber(X)$
			\begin{itemize}
				\item
				\vspace{0.2cm}
				Large well-order: $|\hartogsNumber(X)|\not\le |X|$
				\vspace{0.2cm}
				\item
				Controlled height: $|\hartogsNumber(X)|\le|\Pow^k(X)|$ for some $k$
			\end{itemize}
			\vspace{0.4cm}
			\item
			Develop cardinal arithmetic in the absence of AC
			\vspace{0.4cm}
			\item
			Use GCH to iteratively squeeze in $\hartogsNumber(X)$ and obtain $|X|\le |\hartogsNumber(X)|$
	%		\begin{itemize}
	%			\vspace{0.3cm}
	%			\item
	%			Suitable instance yields $|X|\le|H(X)|$ or $|H(X)|\le |\Pow^{k-1}(X)|$
	%			\vspace{0.3cm}
	%			\item
	%			Iterate in the latter case until $|H(X)|\le |\Pow^{0}(X)|=|X|$
	%		\end{itemize}
		\end{enumerate}
\end{frame}

\begin{frame}{Ordinal Theory}
	content
\end{frame}

\begin{frame}{Hartogs Numbers}
	content
\end{frame}

\begin{frame}{Cardinal Arithmetic}
	content
\end{frame}

\begin{frame}{Iterating GCH}
	content
\end{frame}

\begin{frame}
	\Huge
	\centering
	Observations
\end{frame}

\begin{frame}{Mechanisation Details}
	content
\end{frame}

\begin{frame}{Comparison to Previous Coq Mechanisation}
	content
\end{frame}

\begin{frame}{Open Questions}
	content
\end{frame}



%\begin{frame}[allowframebreaks]{Related Work}
\begin{frame}{Bibliography}
%\scriptsize
\small
\bibliographystyle{apalike}
\bibliography{abstract}
\end{frame}


\end{document}

