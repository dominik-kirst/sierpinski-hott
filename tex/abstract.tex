\documentclass{easychair}
\usepackage{fullpage}
\usepackage{amsmath}
\usepackage{amssymb}
\usepackage{amsthm}
\usepackage{caption}
\usepackage{subcaption}
\usepackage{hyperref}

\title{The Generalised Continuum Hypothesis\\Implies the Axiom of Choice in HoTT}

\author{Dominik Kirst \and Felix Rech}

\institute{Saarland University, Saarland Informatics Campus, Germany}

\authorrunning{Dominik Kirst \and Felix Rech}

\titlerunning{A Toolbox for Mechanised First-Order Logic}

\newcommand{\nat}{\mathbb{N}}
\newcommand{\pow}{\mathcal{P}}
\newcommand{\hprop}{\mathsf{hProp}}
\newcommand{\hset}{\mathsf{hSet}}

\newtheorem{lemma}{Lemma}
\newtheorem{theorem}{Theorem}

\begin{document}

\maketitle

We report on a Coq mechanisation\footnote{\url{https://github.com/dominik-kirst/sierpinski-hott}} of Sierpi\'nski's result~\cite{sierpinski1947hypothese} that the generalised continuum hypothesis (GCH) implies the axioms of choice (AC), implemented using the HoTT Library~\cite{bauer2017hott}.
The result was historically formulated for ZF set theory and in previous work~\cite{kirst2021generalised} we showed that a corresponding statement holds for Coq's constructive type theory.
Without additional axioms, however, Coq's type theory is not perfectly suited to represent set-theoretic results, especially due to the lack of unique choice and extensionality principles.
Since these are all available in HoTT, we now complement the previous Coq development using ad-hoc assumptions with a new mechanisation assuming the univalence axiom and propositional resizing (as well as propositional truncation).
To allow reusing parts of the original Coq code, we opted for the HoTT Library instead of the more deviant Unimath Library~\cite{UniMath}.

\section{Formulating GCH in HoTT}

In classical set theory, Cantor's continuum hypothesis (CH) states that there are no cardinalities strictly between the set $\nat$ of natural numbers and its power set $\pow(\nat)$.
Much more generally, GCH states that there are no cardinalities strictly between any infinite set and its power set, controlling the extent of the power set operation to a considerable degree.

Using the type $\nat:\hset$ of natural numbers, the relation $X\le Y:=\exists f:X\to Y.\, \mathsf{injective}\,f$ for cardinality comparisons, and the operation $\pow(X):=X\to\hprop$ for the power set, we formulate GCH in HoTT as:
$$\forall X Y:\hset.~\nat \le X\le Y\le \pow(X)~\,\to~\, Y\le X ~+~ \pow(X)\le Y$$
By Cantor's theorem in the form refuting injections $\pow(X)\le X$, the concluding disjunction is exclusive and therefore GCH can be seen to be a proposition.
Moreover, employing a refinement of Cantor's theorem, it follows that already a weak formulation of CH without disjunction implies excluded middle, which is a slight strengthening of the connection given in~\cite{bridges2016continuum}:

\begin{lemma}
	$(\forall X:\hset.\,\nat \le X\le \pow(\nat)\to X\not\le \nat\to \pow(\nat)\le X)~\to~\forall P:\hprop.\,P+\neg P$
\end{lemma}
\begin{proof}[Sketch]
	We set $X:=\Sigma\, p:\pow(\nat).\,\mathsf{sing}\,p\lor (P+\neg P)$ where $\mathsf{sing}\,p$ denotes that $p$ is a singleton.
	The comparisons $\nat \le X\le \pow(\nat)$ are obvious and if we assume $X\le\nat$ for the last condition, we may assume $P+\neg P$ due to the negative goal, therefore obtain an injection $\pow(\nat)\le X$ yielding the contradiction $\pow(\nat)\le \nat$.
	Thus we obtain an injection $i:\pow(\nat)\to X$ and by a refinement of Cantor's theorem we can explicitly construct some $p:\pow(\nat)$ such that $\pi_1(i\,p)$ is not a singleton.
	But then $P+\neg P$ must be the case.
\end{proof}

For our main result assuming GCH as a premise, we are therefore able to argue classically where needed.
Moreover, it would make no difference if we were to formulate the conclusion of GCH with a modest-looking implication instead of the constructively (seemingly) stronger disjunction or, employing the classical Cantor-Bernstein theorem, with bijections instead of only the converse injections.


\section{Proof Overview}

The proof that GCH implies AC can be organised into two parts.

\section{Comparison to Previous Mechanisation}

\section{Open Questions}


\bibliographystyle{plain}
\scriptsize{\bibliography{abstract}}
\vspace{-20em}
\end{document}

