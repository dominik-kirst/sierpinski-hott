\documentclass{easychair}
\usepackage{fullpage}
\usepackage{amsmath}
\usepackage{amssymb}
\usepackage{amsthm}
\usepackage{caption}
\usepackage{subcaption}
\usepackage{hyperref}
\usepackage{enumitem}
\usepackage{cleveref}

\title{The Generalised Continuum Hypothesis\\Implies the Axiom of Choice in HoTT}

\author{Dominik Kirst \and Felix Rech}

\institute{Saarland University, Saarland Informatics Campus, Germany}

\authorrunning{Dominik Kirst \and Felix Rech}

\titlerunning{A Toolbox for Mechanised First-Order Logic}

\newcommand{\nat}{\mathbb{N}}
\newcommand{\bool}{\mathbb{B}}
\newcommand{\pow}{\mathcal{P}}
\newcommand{\hprop}{\mathsf{hProp}}
\newcommand{\hset}{\mathsf{hSet}}
\newcommand{\Ord}{\mathsf{Ord}}

\newtheorem{lemma}{Lemma}
\newtheorem{theorem}{Theorem}
\newtheorem{fact}{Fact}
\newtheorem{larry}{Corollary}

\begin{document}

\maketitle

We report on a Coq mechanisation of Sierpi\'nski's result~\cite{sierpinski1947hypothese} that the generalised continuum hypothesis (GCH) implies the axioms of choice (AC), implemented using the HoTT Library~\cite{bauer2017hott}.
The result was historically formulated for ZF set theory and in previous work~\cite{kirst2021generalised} we showed that a corresponding statement holds for Coq's constructive type theory.
Without additional axioms, however, Coq's type theory is not perfectly suited to represent set-theoretic results, especially due to the lack of unique choice and extensionality principles.
Since these are all available in HoTT, we now complement the previous Coq development using ad-hoc assumptions with a new mechanisation assuming the univalence axiom and propositional resizing.
To allow reusing parts of the original Coq code, we opted for the HoTT Library instead of the more disparate Unimath Library~\cite{UniMath}.
The development is available as part of the HoTT Library.

\section{Formulating GCH in HoTT}

In classical set theory, Cantor's continuum hypothesis (CH) states that there are no cardinalities strictly between the set $\nat$ of natural numbers and its power set $\pow(\nat)$.
Much more generally, GCH states that there are no cardinalities strictly between any infinite set and its power set, controlling the extent of the power set operation to a considerable degree.

Using the type $\nat:\hset$ of natural numbers, the relation $X\le Y:=\exists f:X\to Y.\, \mathsf{injective}\,f$ for cardinality comparisons, and the operation $\pow(X):=X\to\hprop$ for the power set, we formulate GCH in HoTT as:
$$\forall X Y:\hset.~\nat \le X\le Y\le \pow(X)~\,\to~\, Y\le X ~+~ \pow(X)\le Y$$
By Cantor's theorem in the form refuting injections $\pow(X)\le X$, the concluding disjunction is exclusive and therefore GCH can be seen to be a proposition.
Moreover, essentially by a refinement of Cantor's theorem, it follows that already a weak formulation of CH without disjunction implies excluded middle (LEM), which is a slight strengthening of the connection given in~\cite{bridges2016continuum}:

\begin{lemma}
	$(\forall X:\hset.\,\nat \le X\le \pow(\nat)\to X\not\le \nat\to \pow(\nat)\le X)~\to~\forall P:\hprop.\,P+\neg P$
\end{lemma}
\begin{proof}[Sketch]
	Given $P:\hprop$, the set $X:=\Sigma\, p:\pow(\nat).\,\mathsf{sing}\,p\lor (P+\neg P)$ satisfies the premises of CH, where $\mathsf{sing}\,p$ denotes that $p$ is a singleton.
	Hence we obtain an injection $i:\pow(\nat)\to X$ and by a variant of Cantor's theorem we obtain some $p:\pow(\nat)$ such that $\pi_1(i\,p)$ is not a singleton, thus $P+\neg P$ must be the case.
\end{proof}

For our main result assuming GCH as a premise, we are therefore able to argue classically where needed.
Moreover, it would make no difference if we were to formulate the conclusion of GCH with a modest-looking implication instead of the constructively (seemingly) stronger disjunction or, employing the classical Cantor-Bernstein theorem, with bijections instead of only the converse injections.




\section{Proof Overview}

It is enough to show that GCH implies the well-ordering theorem (WO), relying on the standard argument that WO implies AC.
For our purposes, we formulate WO as the guarantee that for every set $X$ there is an ordinal $\alpha:\Ord$ with $X\le \alpha$, where $\Ord$ is defined as the type of sets equipped with transitive, extensional, and well-founded relations as in Section 10.3 of the HoTT book.
Following this presentation, we establish the basic facts that isomorphic ordinals are equal, that ordinals satisfy trichotomy (using LEM), that $\Ord$ itself is an ordinal, and that every ordinal is isomorphic to the type of all smaller ordinals.
We also provide constructions of successor and limit ordinals that are not relevant for the main result.
The central ingredient for the main result is that for every set we can construct an ordinal of controlled cardinality:

\newpage

\begin{lemma}
	\label{hartogs}
	Assuming LEM and a set $A:\hset$, we can construct the so-called \emph{Hartogs number} $\aleph(A):\Ord$ on the same universe level as $A$ and satisfying $\aleph(A)\le\pow^3(A)$ as well as $\aleph(A)\not\le A$.
\end{lemma}

\begin{proof}[Sketch]
	Preliminarily, we define $\aleph'(A)$ as the type of ordinals admitting injections into $A$, ordered by the natural ordering.
	This definition increases the universe level but embeds into the resized triple power set $\pow^3(A)$ by mapping every ordinal to its induced partial order of $A$ (relying on trichotomy).
	We then obtain $\aleph(A)$ as the image of this embedding, retain the bound against $\pow^3(A)$ and conclude $\aleph(A)\not\le A$ since otherwise $\aleph(A)$ would be an initial segment of $\aleph'(A)$, although they are isomorphic by construction.
\end{proof}

The remainder of the proof consists of showing that GCH ensures $A\le\aleph(A)$, at least for large enough $A$.
In preparation, we record the necessary amount of cardinal arithmetic phrased for large sets:

\begin{lemma}
	\label{cardinals}
	\setlist{nolistsep}
	\begin{enumerate}[noitemsep]
		\item
		Given LEM, any set $X$ with $\nat\le X$ satisfies $\pow(X)+\pow(X)\simeq \pow(X)$.
		\item
		For sets $X,Y$ with $X+X\le X$ and $\pow (X)\le X+Y$ we obtain $\pow (X)\le Y$.
	\end{enumerate}
\end{lemma}
\begin{proof}[Sketch]
	(1) is by equational reasoning, relying on some non-constructive equivalences such as $\hprop\simeq\bool$, and (2) is by another Cantor-style argument since $X$ essentially cannot contribute to $\pow (X)\le X+Y$.
\end{proof}

We show abstractly that GCH normalises every operation $F$ on sets behaving like the Hartogs number:

\begin{lemma}
	\label{sierpinski}
	Assume GCH and a function $F:\hset\to\hset$ preserving the universe level such that there is $k:\nat$ with $F(X)\le \pow^k(X)$ for and $F(X)\not\le X$ for all $X$.
	Then for every set $X$ we obtain $X\le F\,(\pow(\nat + X))$.
\end{lemma}

\begin{proof}[Sketch]
	We show that already $X':=\pow(\nat + X)$ embeds into $F(X')$ by induction on k.
	The base case $F(X)\le \pow^0(X)$ conflicts with $F(X)\not\le X$ and in the successor case applying GCH on a suitable instance either yields the claim directly or makes the inductive hypothesis applicable, employing the previous lemma.
\end{proof}

So in summary, GCH implies $X\le \aleph\,(\pow(\nat + X))$ for all $X$, hence WO and ultimately AC.


\section{Comparison to Previous Coq Mechanisation}

The previous development~\cite{kirst2021generalised} consists of two mechanisations, one for an axiomatised type of sets and one for Coq's type theory itself.
While the former uses the set-theoretic notion of ordinals for the first part of the proof (\Cref{hartogs}), the latter employs the partial well-orders representable within a given type since type-theoretic ordinals are not available in Coq without further assumptions.
In the HoTT mechanisation, well-behaved ordinals are available and so we could follow the less ad-hoc set-theoretic outline.
As a consequence the HoTT mechanisation is shorter, the parts contributing to the main result amount to roughly 1400loc compared to the 1700loc of the previous type-theoretic mechanisation.

Switching from Coq to HoTT caused a few other notable differences.
Due to univalence, the equational reasoning necessary for \Cref{cardinals} did not rely on setoid rewriting anymore, and the assumptions of functional and propositional extensionality could be eliminated.
Using $\hprop$ instead of Coq's impredicative $\mathsf{Prop}$ universe for logical expressions and the power set operation further eliminated the assumption of unique choice but also introduced the overhead of proving some types to be propositions and resizing some predicates by hand.
Especially resizing the Hartogs number to make \Cref{sierpinski} applicable was intricate and the current solution employing an additional injection $\pow^3(A)\to\pow^3(A)$ to fix the levels is not fully satisfactory.


\section{Open Questions}

We record the following questions we would be interested to investigate and discuss:
\setlist{nolistsep}
\begin{itemize}[noitemsep]
	\item
	Are there generalisations of constructive versions of CH such as the one in~\cite{gielen1981continuum}?
	If so, do constructive versions of GCH imply constructive versions of WO?
	\item
	To what extent can our formulation of GCH be generalised (e.g.\ to non-truncated injections or higher n-types than sets) without becoming inconsistent? Would these principles be meaningful?
	\item
	Could Coq's impredicative $\mathsf{Prop}$ universe be used in context of the HoTT Library to ease impredicative constructions (i.e.\ spare resizing by hand and performance issues with normalisation)?
\end{itemize}


\bibliographystyle{plain}
\scriptsize{\bibliography{abstract}}
\vspace{-20em}
\end{document}

